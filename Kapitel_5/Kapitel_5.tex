\documentclass[a4paper, 12pt]{article} % Präambel
% !TeX encoding=UTF-8
% !TeX spellcheck=de_DE_frami
\usepackage[T1]{fontenc}
\usepackage[utf8]{inputenc}
\usepackage[ngerman]{babel}
\usepackage{marvosym}
% \usepackage{parskip}
\DeclareUnicodeCharacter{20AC}{\EUR}
% \setcounter{secnumdepth}{4}
\usepackage{array}
\usepackage{pifont}
\usepackage{amsmath}

\begin{document}    % Dokument
\begin{center}
\begin{tabular}{l||c<{Text}r}
    \multicolumn{2}{c} {Linksbündig  Zentriert} & Rechtsbündig\\
    Meine       & erste     & Tabelle     \\
\end{tabular}\vspace*{3cm}

\begin{tabular}{|l|p{3.5cm}|m{3.5cm}|b{3.5cm}|}
    \firsthline
    Ausrichtung & oben & mitte & unten\\
    \hline\cline{1-2} \vline Mitte\vline &Dieser Text wird umgebrochen und mit der obersten Zeile
    ausrichtet. &Dieser Text hingegen wird so formatiert, dass er
    im Vergleich mit der Umgebung mittig steht. &
    Schließlich kann man einen Text auch mit der untersten Zeile
    ausrichten.\\\hline
\end{tabular}\vspace*{3cm}

\begin{tabular}{l|>{\itshape}l}
    Deutsch & English\\
    \hline Buch    & Book\\
    Tisch   & Table\\
\end{tabular}\vspace{3cm}

\begin{tabular}{|>{\sffamily}l|*{3}{>{\sffamily}r}|}
    \firsthline
    \textbf{Land} &  \textbf{2002} &  \textbf{2001} &  \textbf{1990}\\
    \hline\hline
    USA & 3 800 000 & 3 778 512 & 3 040 932\\
    VR China & 1 602 156 & 1 421 268 & 618 000\\
    Japan & 935 000 & 932 904 & 857 268\\
    Russland & 888 936 & 888 382 & -\\
    Kanada & 565 000 & 564 108 & 482 028\\
    Frankreich & 549 245 & 545 000 & 419 219\\
    Deutschland & 543 561 & 566 835 & 566 484\\
    \hline\hline
    Weltproduktion & 15 750 000 & 15 684 000 & 11 179 000\\
    \lasthline

\end{tabular}\vspace{3cm}

\begin{tabular}{|>{\ding{212}\ttfamily}l<{\ding{43}}|>{\itshape}p{4cm}|}
    \firsthline
    \multicolumn{2}{|c|}{Class \flqq\texttt{String}\frqq}\\
    \hline charAt & Returns the char value at the specified index\\
    \cline{2-2}codePointAt & Returns the character (Unicode code point)
    at the specified index\\
    \hline
\end{tabular}\vspace{3cm}

\begin{tabular}{|*{4}{>{\sffamily}p{3.2cm}|}}
    \firsthline
    \multicolumn{4}{|c|}{\bfseries\itshape Das alte Italien}\\\hline
    \multicolumn{2}{|c|}{\bfseries Antike} & \multicolumn{2}{c|}{\bfseries Mittelalter}\\\hline
    \multicolumn{1}{|c|}{\itshape Republik}
    & \multicolumn{1}{c|}{\itshape Kaiserreich}
    & \multicolumn{1}{c|}{\itshape Franken} 
    & \multicolumn{1}{c|}{\itshape Teilstaaten}\\\hline
    In den Zeiten der römischen Republik standen dem Staat jeweils zwei Konsuln vor,
    deren Machtbefugnisse identisch waren.
    & Das römische Kaiserreich wurde von einem Alleinherrscher, dem
    Kaiser regiert
    & In der Völkerwanderungszeit übernahmen die Goten und später die Franken
    die Vorherrschaft 
    & Im späteren Mittelalter regierten Fürsten einen Flickenteppich von Einzelstaaten\\
    \lasthline
\end{tabular}\vspace{3cm}

\begin{tabular}{||>{\bfseries}l|ll|>{\sffamily\slshape}m{2cm}>{\sffamily\slshape}m{3cm}|r||}
    \firsthline
    & \multicolumn{2}{c|}{Herkunft} & \multicolumn{2}{c|}{Beschreibung}& \\\cline{2-5}
    Malt Whisky & Region & District & Colour & Nose & Pkt\\\hline\hline
    Lagavulin & Islay & South Shore & Full Amber
    & Sea-salt, peat, intense dryness, sherry & 95\\\cline{4-6}
    The Macallan & Highlands & Speyside & Amber
    & Sherry, and buttery, honeyish, malt character & 87\\\cline{4-6}
    Talisker & Highlands & Skye & Amber-red, bright
    & Pungent, smoke-accented, rounded & 90\\\cline{4-6}
    \lasthline
    
\end{tabular}\vspace{3cm}

\begin{tabular}[b]{p{3cm}}
    \firsthline Spalten werden mit zusätzlichem Leerraum
    auf der linken und rechten Seite gesetzt.\\\lasthline
\end{tabular}
\begin{tabular}{p{3cm}}
    \firsthline Wie man hier sieht, kann man diesen Leerraum
    auch entfernen.\\\lasthline
\end{tabular}
\begin{tabular}[t]{!{\ding{116}}p{3cm}!{\ding{116}}}
    \firsthline Es lassen sich statt vertikaler Linien auch andere
    Zeichen zur Gestaltung der Spaltentrennungen verwenden.\\\lasthline
\end{tabular}\vspace{3cm}

\begin{tabular}{c r@{,}l}
    \hline Ausdruck &\multicolumn{2}{c}{Wert}\\\hline
    $\pi$&    3&1416\\$\sqrt{55555}$    &  235&70\\$\exp(8)$        & 2981&0\\\hline
\end{tabular}\vspace{3cm}

\begin{tabular}{|l|c!{\ding{86}}r@{:}l|r@{,}l@{ E}l@{ V }|}
    \firsthline
    \multicolumn{7}{|c|}{\bfseries Messbericht}\\\hline
    Anordnung & \# &  \multicolumn{2}{c|}{Verhältnis} & \multicolumn{3}{c|}{Messwert}\\\hline
    Aufbau A & 1 & 10 & 1 & 1 & 1234 & -4 \\
    & 2 & 7 & 2 & 12 & 142 & -2 \\
    & 3 & 2 & 8 & 4 & 2 & +120 \\\hline

    Aufbau B & 1 & 100 & 1 & 1 & 0 & -9999 \\
    & 2 & 1 & 100 & 2 & 777432 & -12 \\
    & 3 & 1 & 200 & 3 & 113 & -2 \\\lasthline
\end{tabular}\vspace{3cm}

\begin{tabular}{|p{3cm}|>{\raggedright\arraybackslash}p{3cm}|%
    p{3cm}|%
    >{\centering\arraybackslash}p{3cm}|}
    \hlineÜblicherweise werden Tabelleneinträge fester Breite in
    Blocksatz gesetzt & \raggedright\arraybackslash Gerade bei schmalen Spalten ist aber oft Flattersatz
    optisch angenehmer &Das geht natürlich auch mit rechtsbündiger Formatierung
    &Auch die Zentrierung einer Tabellenspalte ist möglich\\\hline
\end{tabular}
\end{center}

Hier zitieren wir die Tabelle~\ref{tab:tabellegleit} im laufenden
Text. Diese Tabelle befindet sich auf
Seite~\pageref{tab:tabellegleit}
\begin{table}[b]
    \begin{tabular}{>{\itshape}lll}
        \hline\hline\bfseries physikalische Größe & 
        \bfseries englische Bezeichnung &
        \bfseries französische Bezeichnung\\\hline
        \hline Absorptionsgrad & absorptance & facteur d’absorption\\
        Arbeit & work & travail\\Dichte & density & masse volumique\\
        Kinetische Energie & kinetic energy & énergie cinétique\\
        Schallleistung & sound energy flux & flux d’énergyacoustique\\\hline
    \end{tabular}%
    \caption{Beispielstabelle mit der Übersetzung von physikalischenGrößen}\label{tab:tabellegleit}
\end{table}
\end{document}