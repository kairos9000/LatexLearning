\documentclass[a4paper, 12pt]{article} % Präambel
% !TeX encoding=UTF-8
% !TeX spellcheck=de_DE_frami
\usepackage[T1]{fontenc}
\usepackage[utf8]{inputenc}
\usepackage[ngerman]{babel}
\usepackage{marvosym}
% \usepackage{parskip}
\DeclareUnicodeCharacter{20AC}{\EUR}
% \setcounter{secnumdepth}{4}
\usepackage{pifont}
\usepackage{enumerate}

\begin{document}    % Dokument

\begin{center}Ein zentrierter Text bleibt auch über mehrere
    Zeilen und Absätze hinweg zentriert.\\
    Umbrüche können beliebig hinzugefügt werden.\end{center}
\begin{flushleft}Bei linksbündigen Flattersatz versucht \LaTeX\
    nicht, einen Ausgleich auf der rechten Seite vorzunehmen.
    Besonders bei sehr schmalen Absätzen wird dadurch ein
    ansprechender Satzspiegel erzeugt.\end{flushleft}
\begin{flushright}Der rechtsbündige Flattersatz ist das Gegenstück
    zum linksbündigen Flattersatz. Man erhält auch bei einem\\
    Umbruch das erwünschte Verhalten.\end{flushright}
\par{\centering Ein zentrierter Text bleibt auch über mehrere
Zeilen und Absätze hinweg zentriert.\\
Umbrüche können beliebig hinzugefügt werden.\par}
{\raggedright Bei linksbündigen Flattersatz versucht \LaTeX\
nicht, einen Ausgleich auf der rechten Seite vorzunehmen.
Besonders bei sehr schmalen Absätzen wird dadurch ein
ansprechender Satzspiegel erzeugt.\par}
{\raggedleft Der rechtsbündige Flattersatz ist das Gegenstück
zum linksbündigen Flattersatz. Man erhält auch bei einem\\
Umbruch das erwünschte Verhalten.\par}

\begin{center} {\large\itshape Einladung zur Universitätsfeier} \end{center}

\begin{flushright} \textsf{Diebis 33\\
    92263 Ebermannsdorf\\
    Deutschland}
\end{flushright}

\begin{center}
    \begin{verse}
\textsc{Zweiter Jäger:} \textbf{Seht mir!} das ist ein \textbf{\large wackrer} 
\textsl{Kumpan}!
\textit{Sie begrüßen ihn.}\\
\textsc{Bürger:} {\Large Oh!} \texttt{\textbf{laßt} ihn!} Er ist 
\textsf{guter {\large Leute}} Kind. \\
\textsc{Erster Jäger:} Wir \emph{auch} nicht auf der \textit{\Large \textbf{Straße}}
gefunden sind. \\
\textsc{Bürger:} {\tiny Ich sag euch,} er \emph{hat} \textsf{\textbf{Vermögen und Mittel.}}
\textbf{Fühlt \large{her}},
das feine \texttt{Tüchlein} am \textsc{\Large Kittel}{\Huge !}\\
\textsc{Trompeter:} Des \emph{Kaisers} 
\textsl{Rock} ist der {\tiny\bfseries höchste} Titel.\\
\textsc{Bürger:} Er erbt eine \emph{\tiny kleine}\textbf{Mützen}
fabr{\Large i}{\Huge k}.\\
\textsc{Zweiter Jäger:} Des Menschen \texttt{Wille},das
 ist sein \textsf{Glück}.\\
 \textsc{Bürger:} Von der {\Large Groß}\textit{mutter}eine
  Kram \emph{und} Laden.\\
  \textsc{Erster Jäger:} \textbf{Pfui!} wer handelt
  mit \textit{Schwefel}faden!\\
  {\large\textsc{Bürger:} Einen \emph{Weinschrank}
  {\small dazu von seiner} Paten;}
 Ein \textbf{Gewölbe} mit zwanzig Stückfaß\texttt{\bfseries Wein}.\\
  \textsc{Trompeter:} Den \textbf{teilt} er mit
  seinen \emph{Kameraden}.\par\bigskip
    \end{verse}
\end{center}

\begin{center}
    \flqq \textbf{\large Maria Stuart}\frqq\\\
    \flqq \textbf{\large Friedrich Schiller}\frqq\
    \begin{verse}
        
\textsc{Elisabeth\footnote{Königin von England}} \emph{ungeduldig}: \textsf{Ich will, daß dieser 
unglücksel'gen Sache}
Nicht mehr gedacht werden soll werden, daß ich endlich\footnote{Es ist \emph{keine} 
gute Idee, die Königin von England zu verärgern.}
\texttt{Will Ruhe davor haben und auf ewig.}\par\bigskip
\end{verse}
\end{center}


\begin{enumerate}
    \item erster Punkt
    \item Zweiter Punkt
    \item Dritter Punkt
    \begin{enumerate}
        \item Unterpunkt
        \begin{enumerate}
            \item Punkt
            \begin{enumerate}
                \item Punkt
            \end{enumerate}
        \end{enumerate}
    \end{enumerate}
\end{enumerate}

\begin{description}
    \item[Punkt] erster Punkt
    \item[Punkt 2,] Zweiter Punkt
    \item[Für Lexikoneinträge] Dritter Punkt
    \begin{description}
        \item[eigene Punkte] Unterpunkt
        \begin{description}
            \item[automatisch fett] Punkt
            \begin{description}
                \item[letzter Punkt] Punkt
                \begin{description}
                    \item[Tiefe nicht auf 4 beschränkt] Punkt
                \end{description}
            \end{description}
        \end{description}
    \end{description}
\end{description}

\begin{itemize}
    \item Mathematik 1
    \begin{itemize}
        \item Lineare Algebra
        \item Analysis
    \end{itemize}
    \item Stochastik
    \begin{itemize}
        \item Grundlagen der Stochastik
        \item Hypothesentests
        \item Kombinatorik
    \end{itemize}
\end{itemize}

\begin{enumerate}
    \item Ein Roboter darf kein menschliches Wesen (wissentlich)
     verletzen oder durch Untätigkeit (wissentlich) zulassen, dass
      einem menschlichen Wesen Schaden zugefügt wird.
    \item Ein Roboter muss den ihm von einem Menschen gegebenen
     Befehlen gehorchen – es sei denn, ein solcher Befehl
      würde mit Regel eins kollidieren.
    \item Ein Roboter muss seine Existenz beschützen, solange dieser 
    Schutz nicht mit Regel eins oder zwei kollidiert.
\end{enumerate}

\begin{enumerate}[A)]
    \item Punkt auf der ersten Ebene
    \item Die Nummerierung erfolgt alphanumerisch mit Klammer
    \begin{enumerate}[{Beispiel} i:]
        \item Punkt auf der zweiten Ebene
        \item Hier haben wir römische Zahlen gewählt
        \item Man beachte, dass Zusatztext geklammert werden muss.
    \end{enumerate}
\end{enumerate}

Belisar mußte diese Brücke zerstören, wenn er Truppen und Getreide
in die Stadt bringen wollte. Er wartete noch einige Zeit auf die
Ankunft des Johannes, aber diesem kühnen General hatten die
Goten\marginpar[linke Goten] {rechte Goten} den Weg verlegt. Er forderte Bessas in der  % Option von \marginpar[left]{right} steht für
                                                                                        % die Seiten auf denen es ausgeben wird bei Büchern
                                                                                        % falls die Randbemerkung auf einer ungeraden Seite steht
                                                                                        % wird right benutzt, falls sie auf einer geraden Seite steht
                                                                                        % wird left benutzt
Stadt auf, einen gemeinschaftlichen Angriff auf das gotische
Lager zu machen, aber der Befehlshaber regte sich nicht, und die
Besatzung lag starr und müßig auf den Wällen Roms.

\verb+\verb!\begin{verbatim*}!\textit{Quelltext}\verb!+
    \textsf{\tiny schwierig\\ oder?}
    \verb+\end{verbatim*}!+


\begin{tabbing}
    \scshape Dramatis \= \hspace*{5cm} \=\kill  % Der Tabulatorstopp nach Dramatis
                                                % ist 5cm lang
                                                % alternativ geht auch:
                                                % \scshape Dramatis \=The Cardinal of Lorraine, \=\kill
    \scshape Dramatis \=\scshape Personae\\[2mm]
    \> Chorus\\
    \> Dr.~John Faustus \\
    \> Wagner, \>\itshape his servant, a student\\
    \> Valdes, \>\itshape his friend\\
    \> Cornelius, \>\itshape his friend\\
    \> The Cardinal of Lorraine, \>\itshape a french priest
\end{tabbing}

\begin{tabbing}
    \scshape Dramatis \=The Cardinal of Lorraine, \=\kill
    \scshape Dramatis \=\scshape Personae\+\\[2mm] % Durch \+ muss nicht vor jede Zeile \> geschrieben werden
    Chorus\\
    Dr.~John Faustus\\
    Wagner, \>\itshape his servant, a student\\
    Valdes, \>\itshape his friend\\
    Cornelius, \>\itshape his friend\\
    The Cardinal of Lorraine,\>\itshape a french priest
\end{tabbing}

\begin{tabbing}
    Package \=\sffamily java.lang\=\sffamily String\=\hspace*{5mm}\=\kill
    Package \>\sffamily java.lang\\
    Class \>\> \sffamily String\\
    Method \>\>\> \sffamily charAt\+\+\+\\
    \>\itshape Returns the char value at the specified index.\\
    \sffamily codePointAt\\
    \>\itshape Returns the character (Unicode code point)\\
    \>\itshape at the specified index.\\
    \<\<\< Class \>\>\sffamily Math\\
    \<\<\< Method \>\>\>\sffamily sin\\
    \>\itshape Returns the trigonometric sine of an angle.\\
    \sffamily cos\\
    \>\itshape Returns the trigonometric cosine of an angle.\\
\end{tabbing}
\end{document}