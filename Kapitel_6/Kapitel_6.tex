\documentclass[a4paper, 12pt]{article} % Präambel
% !TeX encoding=UTF-8
% !TeX spellcheck=de_DE_frami
\usepackage[T1]{fontenc}
\usepackage[utf8]{inputenc}
\usepackage[ngerman]{babel}
\usepackage{marvosym}
\usepackage{parskip}
\DeclareUnicodeCharacter{20AC}{\EUR}
% \setcounter{secnumdepth}{4}
\usepackage{array}
\usepackage{pifont}
\usepackage{amsmath}
\usepackage{calc}

\newcounter{zahlA}%   in der Präambel
\newcounter{zahlB}%   in der Präambel
\newlength{\templaenge}%   in der Präambel
\newlength{\tunneldiode}
\newlength{\avalancheDiode}
\newlength{\gunnDiode}

\begin{document}    % Dokument

\begin{enumerate}\item Zunächst normale Zählung. Jetzt erhöhen wir 
    \texttt{enumi}um 5.\addtocounter{enumi}{5}\item
    Man beachte, dass der Zähler um 6 gestiegen ist, da die normale
    Erhöhung auch hinzukommt. Man kann auch rückwärts springen.
    \addtocounter{enumi}{-10}\item So kann man auch mal negativ aufzählen.
     Der nächste Punkt
     soll (ungefähr) die Seitenzahl erhalten.
     \setcounter{enumi}{\value{page}-1}
     \item Hier ist (vielleicht) die Seitenzahl. 
     Abweichungen ergeben
     sich, da erstens hochgezählt wird und zweitens bei Auslesen der
     Seitenzahl \LaTeX\ vielleicht noch nicht umgebrochen hatte.
\end{enumerate}

\setcounter{zahlA}{10}
\setcounter{zahlB}{70}

Ich habe \arabic{zahlA}\EUR~in der Hand. Ein Römer würde vielleicht
\Roman{zahlA} Sesterzen haben. Wenn ich \arabic{zahlB}\EUR~hinzunehme,
 \addtocounter{zahlA}{\value{zahlB}}so 
 habe ich \arabic{zahlA}\EUR~in der Hand.Ein Römer hätte entsprechend
 \Roman{zahlA} Sesterzen.

 \setcounter{zahlA}{4}Ich habe \refstepcounter{zahlA}\label{euroinderhand}%
 \arabic{zahlA}\EUR~in der Hand.\par Später werde ich mich erinnern, dass ich 
 \ref{euroinderhand}\EUR~in der Hand hatte und zwar auf Seite 
 \pageref{euroinderhand}.


%  \renewcommand{\thezahlA}{(\roman{zahlA})}%
%  \makeatletter\renewcommand{\p@zahlA}{\theparagraph~}\makeatother% to make @ a letter, so that it is not recognized by the latex compiler
 In diesem Handbuch werden
 \refstepcounter{zahlA}\thezahlA\ \LaTeX-Eigenschaften vorgestellt,
 \refstepcounter{zahlA}\thezahlA\label{zweitens} mit Beispielen
 illustriert und
 \refstepcounter{zahlA}\thezahlA{} durch Übungen verinnerlicht.\par
 Der Hinweis auf Beispiele war in Abschnitt~\ref{zweitens} auf
 Seite~\pageref{zweitens}.

 \makeatletter\p@zahlA\thezahlA

 Genau ein\hspace{1in}Zoll Abstand.


 A\hfill B\hspace{\stretch{1}}C\\
 Anfang\hfill Mitte\hfill Ende\\
 Anfang\dotfill Mitte\hrulefill Ende\\
 Anfang\dotfill\hfil\hrulefill Ende\\
 Anfang\dotfill\hfill\hrulefill Ende\\
 Anfang\dotfill\hspace{0pt plus 1filll}\hrulefill Ende


\setlength{\templaenge}{8cm}
Abstand\hspace{\templaenge}von \verb+\templaenge+\\
\settowidth{\templaenge}{\textbf{Rumpelstilzchen}}
Ach, wie gut, dass niemand weiß, dass ich \textbf{Rumpelstilzchen} heiß’.\\
Ach, wie gut, dass niemand weiß, dass ich \hspace{\templaenge} heiß’.

\settowidth{\tunneldiode}{\textbf{\Large Tunneldiode}}
\settowidth{\avalancheDiode}{\textbf{\Large Avalanche-Diode}}
\settowidth{\gunnDiode}{\textbf{\Large Gunn-Diode}}


\begin{tabular}{|p{\tunneldiode}|p{\avalancheDiode}|p{\gunnDiode}|}

    \textbf{\Large Tunneldiode} & \textbf{\Large Avalanche-Diode} & 
    \textbf{\Large Gunn-Diode}\\
    {\small\sffamily William Shockley äußerte bereits 1954 die Vermutung, dass in 
    bestimmten Werkstoffen die Migrationsgeschwindigkeit der Elektronen unter
    Einfluss hoher elektrischer Feldstärken abnimmt.} & 
    {\small\sffamily Bei der Avalanche-Diode tritt eine lawinenartige Verstärkung des 
    Zenereffekts auf. Die durch den Zenereffekt freigesetzten 
    Ladungsträger werden bei ausreichender Feldstärke und 
    Migratonslänge derart beschleunigt, dass ihre kinetische Energie 
    ausreicht, um sekundäre Elektronen vom Valenzband in das 
    Leistungsband übergehen zu lassen.} & 
    {\small\sffamily Im Jahr 1963 entdeckte der brititsche Physiker 
    Jan B. Gunn bei seinen Untersuchungen im IBM-
    Forschungslabor an GaAs- und InP-Widerständen,
    dass deren Eigenrauschen ab bestimmten Spannungen stark
    anstieg, bis hin zur Oszillation.}

\end{tabular}\vspace{3cm}

\setlength{\arrayrulewidth}{1pt}

\begin{tabular}%
    {|*{4}{p{(\linewidth-\tabcolsep*8-\arrayrulewidth*5)/4}|}}  % tabcolsep is the distance of the columns from each other
                                                                % and the border of the table
                                                                % arrayrulewidth is the width of the separating lines for the columns
    \hline\multicolumn{4}{|c|}{\bfseries
    Tabelle mit Textbreite und vier gleichbreiten Spalten}\\
    \hline\itshape
    Spalte 1 & \itshape
    Spalte 2 &\itshape
    Spalte 3 & \itshape
    Spalte 4\\
    \hline\hline
    Dieser Text ist in der ersten Spalte &
    Dieser Text ist in der zweiten Spalte &
    Dieser Text ist in der dritten Spalte &
    Dieser Text ist in der vierten Spalte\\
    \hline
\end{tabular}

\end{document}