\documentclass[a4paper, 12pt]{article} % Präambel
% !TeX encoding=UTF-8
% !TeX spellcheck=de_DE_frami
\usepackage[T1]{fontenc}
\usepackage[utf8]{inputenc}
\usepackage[ngerman]{babel}
\usepackage{marvosym}
% \usepackage{parskip}
\DeclareUnicodeCharacter{20AC}{\EUR}
% \setcounter{secnumdepth}{4}

\begin{document}    % Dokument

\title{Unser erster Bericht}
\author{Hans Müller \and Sebastian Meier}
\date{23. Februar 2021}
\maketitle
\section*{Zusammenfassung}

Dies ist das erste strukturierte Dokument, welches innerhalb der Übungen
geschrieben wird. Hier wird der Abschnitt \dq Nummerierung\dq{} zitiert.
Der Abschnitt hat die Nummer~\ref{nummerierung} und ist 
auf der Seite~\pageref{nummerierung}
\tableofcontents

\section{Beschreibung}
\subsection{Aufgabe}\label{aufgabe}
Schreiben Sie ein Dokument als Artikel mit einer nicht-nummerierten
Zusammenfassung. Speichern Sie diesen Text in einer Datei.

\subsection{Deutscher Text}
Achten Sie darauf, dass Sie in die Präambel alle Einstellungen für 
deutsche Texte einfügen.

\section{Arbeitsschritte}

\section{Inhaltsangabe}
Das Dokument soll eine automatische Inhaltsangabe erhalten, die nach der 
Zusammenfassung auszugeben ist.

\section{Unterstrukturen}
Hier ist die oberste Gliederungsebene für Artikel.

\subsection{Nummerierung}\label{nummerierung}
Die Nummerierung erfolgt automatisch und hierarchisch.

\subsubsection{Änderungen von Nummerierung}
Es sind keine manuellen Änderungen der Nummerierung notwendig. 
Neu eingeschobene Abschnitte werden automatisch korrekt nummeriert.

\paragraph{Paragraphen} Paragraphen werden zwar intern nummeriert, aber in der
Standardeinstellung werden diese Nummern nicht ausgegeben und die 
Paragraphen werden in der Standardeinstellung auch nicht in das 
Inhaltsverzeichnis gestellt.

\subsubsection{Referenzen auf Nummern}
In der Vorlesung wird gezeigt, wie man auch logische Gliederungsnummern 
durch Querbezüge verweisen kann. Hier ist eine Referenz auf die Aufgabenstellung:
Die Aufgabenstellung hat die Nummer~\ref{aufgabe} und die Seite~\pageref{aufgabe}

\section{Formatierung}
\subsection[Lange Überschriften]{Auch sehr lange Überschriften können verwendet werden, wobei hier
einer Kurzfassung ins Inhaltsverzeichnis geschrieben werden soll}
Blicken Sie in das Inhaltsverzeichnis.

\subsection{Fettdruck}
Größe und Fettdruck von Überschriften wird automatisch eingestellt.

\subsection{Abstände}
Beachten Sie, wie Abstände automatisch gesetzt werden.

\subsection{Querbezüge} \label{subsec:referenzen}
Zu den besonderen Stärken von \LaTeX\ zählt die Verwendung von
Querbezügen innerhalb eines Dokuments.
\dots
Um den Unterabschnitt zu zitieren, in dem sich dieses Beispiel
befindet, schreibt man Abschnitt~\ref{subsec:referenzen}, wobei
das zugehörige Label an den Strukturbefehl angefügt wird.
Dieses Label befindet sich auf der Seite~\pageref{subsec:referenzen}.

\hspace{\fill} Dies ist ein Absatz, der eine Einrückung am Anfang besitzt.
Der nächste Absatz jedoch soll nicht eingerückt werden.\par
Hier kommt ein nicht eingerückter Absatz, der aber mit
dieser Ausnahme sonst keine Besonderheiten aufweist.\par
\vfill
Beim nachfolgenden Absatz wirkt die Aufhebung der
Einrückung nicht mehr, d.\,h.\ er wird wieder nach
Standard eingerückt.\par

Im ersten Absatz springen wir \hfill zum Ende \linebreak
und fahren in der zweiten Zeile fort. Hier lassen wir den Absatz langsam 
ausklingen.\par\bigskip

Nachdem wir ungefähr eine Zeile auslassen, beginnt unser zweiter Absatz. Hier
lassen wir zunächst horizontal einen Zentimeter \hspace*{1cm} Platz, der
handschriftlich gefüllt wird. Ebenso werden sieben Millimeter\vspace{7mm}

vertikaler Raum innerhalb des Absatzes hinzugefügt. Damit endet der
zweite Absatz.\par

Im dritten Absatz steht nicht mehr viel.\par\bigskip


Aus \flqq Lieder\frqq\ des Heinrich von Morungen (\textasteriskcentered 
~12.~Jahrhundert, \textdagger ~13.~Jahrhundert) stammt \flqq V. Von den Elben\frqq\
in mittelhochdeutscher Sprache:\par\medskip\noindent
\glqq Von den elben wirt entsehen vil manic man,\\
\hspace*{2ex} s\^{o} bin ich von gr\^{o}zer lieben ents\^{e}n \\
von der besten, die \'{\i}e dehein m\'{a}n ze virunt gewan.\\
\hspace*{2ex} wil aber s\^{\i} dar \'{u}mb\`{e} mich v\^{e}n,\\
\hspace*{2ex} Und ze unstaten st\^{e}n,\\
\hspace*{4ex} mac si danne rechen sich\\
\hspace*{4ex} und tuo, des ich si bite. s\^{o} vr\'{e}ut si
s\^{o} s\^{e}\textquoteright re mich,\\
\hspace*{2ex} daz m\^{\i}n l\^{\i}p vor wunnen muoz zerg\^{e}n.\grqq\par\bigskip



Ein mittelenglisches \frq Spottlied auf Richard von Cornwall\flq\ 
\textbraceleft A Song of Lewes\textbraceright\
beginnt wie folgt:\par\medskip\noindent
\glqq Sitte\th alle stille \&\ herkne\th to me!\\
\th e kyng of alemaigne, bi me leaute,\\
\th ritti \th ousent pound~(\textsterling)~askede he\\
fforte make \th e pees in \th e countre,\\
\hspace*{4ex} ant so he dude more.\\
\hspace*{2ex} Richard, \th ah \th ou be euer trichard,\\
\hspace*{4ex} trichhen shalt \th ou neuermore!\grqq
\end{document}
